\begin{PreChapter}{介\quad 绍}
	动力系统的一般理论起源于常微分方程, 其基础由H. Poincaré(1854–1912)和A.M. Lyapunov(1857–1918)奠定. G.D. Birkhoff(1884–1944)对该理论作出了重要贡献, 他是“动力系统”一词的提出者, 并且利用了拓扑的方法, 在抽象的层次很大程度的发展了动力系统理论. 动力系统的概念是一般的科学上的演化(依赖于时间)过程概念的数学化, 这些过程可以是相当不同的自然现象. 动力系统自然地诞生于对物理、化学、生物、生态、经济, 甚至社会现象的研究. 动力系统的概念包括一个可能的状态的集合(相空间), 以及状态关于时间的演化法则. 因此, “动力系统”这一概念覆盖了相当多模型的种类, 只要这些模型也许能描述任意对象关于时间的演化和依赖于时间的过程. 例如, 这些对象和过程包括由在连续介质力学和数学物理中产生的非线性演化偏微分方程生成的模型. 这些模型需要无穷维空间用以表示各种可能的状态. 在本书中, 我们专注于(无穷维)系统, 这些系统展示了各种类型的能量迁移和耗散. 似乎(见, 例如\cite{Hale88},\cite{Raugel02},\cite{Temam97}中的讨论)这些作用在\cite{Levinson45}的文章中被严格定义, 该文章以现代(数学)的形式介绍了(动力学的)耗散性的概念;也可见\cite{Billoti71},\cite{Coddington55},\cite{Pliss66},\cite{Pliss77}. 耗散性意味着动力学行为在相空间中被局部化. 这可以被表达为关于有界吸收集存在性的陈述. 在具有有限自由度的系统的情形中, 这种局部化允许我们选择限制性对象, 例如吸引子, 它含有关于系统稳定性的重要信息. 无穷维系统的情况会变得非常不同. 为了挑选出相应的限制机制, 我们需要额外的演化的紧致性条件. 这使得无穷维的理论变得更加复杂. 尽管如此, 到目前为止, 集中关注不同类型的PDE模型, 无穷维动力系统理论的几个重要的方面已经得到了发展(见, 例如这些专著\cite{Babin92},\cite{Chepyzhov02},\cite{Chueshov99},\cite{Chueshov10},\cite{Chueshov08},\cite{Hale88},\cite{Ladyzhenskaya91},\cite{Robinson01},\cite{Sell02},\cite{Temam97}, 以及这些研究\cite{Babin06},\cite{Miranville08},\cite{Raugel02}). 
	
	本书聚焦于无穷维耗散动力系统. 为了将理论的一般化达到合理的程度, 我们的考虑是相对抽象的, 并且协调了各种在抽象空间上定义的一般的演化方程. 我们的目标是介绍与基本的动力系统长时间行为有关性质的一般方法和抽象结果. 我们的主要工具是基于耗散系统相关的拟稳定性质. 粗略的讲, 拟稳定性意味着我们可以通过将两条轨道的差异分解为收敛部分和紧致部分来控制轨道的发散. 
	
	本书的主要特点(相较其他书籍)有以下几个方面:

	\begin{itemize}
		\item 我们展示、发展和阐述了一种基于相对较新的观察的动力系统紧性的方法, 它在\cite{Khanmamedov06}被提出(亦可见\cite{Chueshov08}和\cite{Chueshov10}), 并且已经被证明在临界非线性问题的研究中非常有效. 这种方法在势能的辅助下, 作为二阶演化方程的补偿紧性方法出现, 并且已经被应用到许多其他情形中. 事实上, 这种方法展示了一种拟稳定性的弱形式. 
		
		\item 为了研究关于有限维吸引子和它的光滑性的问题, 我们促成和发展了一种关于一些二阶演化方程拟稳定性的全新方法, 它最初在\cite{Chueshov04}中被介绍(亦可见\cite{Chueshov08}和\cite{Chueshov10}以及最近的研究\cite{Chueshov13}). 这种方法的主要优势是可以将对动力系统的初始的光滑性的要求降到最低. 
		
		\item 在本书中, 我们非常注重将理论应用于无限维系统, 这些系统源自于连续介质力学和数学物理. 然而, 为了使抽象方案更加透明并呈现复杂行为的不同可能场景, 我们集中使用了低维的ODE作为例子, 其中一些是真实世界PDE模型的低程度的近似. 
		
		\item 我们提供了适用于无穷维(局部紧)情形的现代形式的动力系统理论的基本概念, 部分素材我们以练习题的形式给出. 其中一部分练习题提供了有关所考虑对象的额外信息, 这使得文本更加集中, 事实上, 许多练习题可以“练习”的标题替换成例如“容易见得”这样的文本. 但是, 我们更倾向于使他们保持“量化”的形式, 并且我们相信, 这会使得内容更加友好. 
	\end{itemize}

	本书的组织方式如下:
	
	\rcpt{cpt:1}针对动力系统的抽象理论中的基本概念和定义. 在本章中, 我们解释了例如轨道和$\omega-$极限集的概念,我们同样介绍了稳定性的一些概念(Lagrange, Poisson和Lyapunov). 我们讨论了独立轨道的可能类型的行为, 例如游荡和非游荡, 循环, 几乎循环和几乎周期运动. 我们主要继承经典材料的表述, 例如\cite{Nemytskii60}和\cite{Sibirsky75}. 在本章中, 我们也介绍了$1$维连续动力系统的完整理论, 并且讨论了Poincaré-Bendixson理论, 该理论介绍了连续时间的$2$维系统主要的定性动力学类型. 总之, 我们通过例子考虑了分歧理论的要素. 

\end{PreChapter}