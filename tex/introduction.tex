\begin{introduction}
	动力系统的一般理论起源于常微分方程,其基础由H.Poincaré(1854–1912)和A.M. Lyapunov (1857–1918)奠定. G.D. Birkhoff (1884–1944)对该理论作出了重要贡献,他是“动力系统”一词的提出者,并且很大程度地利用了拓扑的方法,将动力系统理论发展到了抽象的高度。动力系统的概念是一般的科学上的演化(关于时间)过程概念的数学化,他们可以是相当不同的自然现象。动力系统自然的诞生于对物理、化学、生物、生态、经济,甚至社会现象的研究。动力系统的概念包括一组可能的状态(相空间),以及状态关于时间的演化法则
	
\end{introduction}