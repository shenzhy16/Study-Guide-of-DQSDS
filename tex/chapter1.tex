%# -*- coding: utf-8-unix -*-
%%==================================================
\chapter{基本概念}

本章收集了一般的动力系统理论中的基本定义、概念和最简单的说明性称述。我们还描述了所有1维和2维连续动力系统的可能场景,并通过例子,讨论了主要的分歧图像。我们后半部分叙述的主要目的是,给读者以低维(1或2维)的连续演化时间算子会产生什么样动力学行为的感觉.

我们主要遵循给出的表达,并且依赖经典的常微分材料;见\cite{Nemytskii60}

\section{演化算子和动力系统}
如同介绍中已经提及的,动力系统的概念包括可能出现的状态的集合(相空间)和状态关于时间的演化法则。之后的叙述中,我们选取完备的度量空间作为相空间,我们记$\mathbb{T}_{+}$为$\mathbb{T}$上的非负元素,其中$\mathbb{T}$为$\mathbb{R}$或$\mathbb{Z}$,用以代表时间。
\begin{defination}{演化算子}{}\index{演化算子}
	设$X$是完备的度量空间,$\mathbb{T}=\mathbb{Z}$或$\mathbb{R}$,对任意$t\in\mathbb{T}$,$S_{t}:X\to X$是连续映射,并且满足半群性质,即:$$S_{0}=id_{X},\quad S_{t+\tau}=S_{t}\circ S_{\tau},\quad\forall t,\tau\in\mathbb{T}^{+},$$则称$\{S_{t}\}_{t\in \mathbb{T}^{+}}$为演化算子(或演化半群、半流)。
\end{defination}