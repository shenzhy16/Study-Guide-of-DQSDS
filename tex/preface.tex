\begin{PreChapter}{原作前言}
	本书的主要目的是介绍无穷维演化模型的背景和最新发展的数学研究方法,考虑各种形式和来源的耗散性和稳定性。
	
	耗散系统的主要特征是存在更高阶的衰减这种能量重新分配机制,这种机制可以导致系统中复杂的极限状态和结构的产生,它在某些情况下是稳定的。人们普遍认为,20世纪80年代在寻找合适的能够解释湍流现象的数学模型的过程中,耗散系统的一般理论得到了显著发展。至今,无穷维耗散系统的研究已经取得了显著进展(见例如以下专著\cite{Babin92},\cite{Chepyzhov02},\cite{Hale88},\cite{Ladyzhenskaya91},\cite{Robinson01},\cite{Sell02},\cite{Temam97}和它们的参考资料)。
	
	相较于其他提及的材料,本书的主要特点是系统性的介绍、发展和利用了拟稳定的方法,这种方法是和Irena Lasiecka在\cite{Chueshov08},\cite{Chueshov10}(亦可见最新的研究\cite{Chueshov13})中合作的,它最初被设计用于二阶非线性阻尼模型。这里我们大幅扩展了这个方法,更多种类的二阶方程、抛物模型以及迟滞偏微分系统也将被包括在考虑范围之内。
	
	我们希望本书不仅对对动力系统的一般理论感兴趣的数学工作者有帮助,也可以对对连续介质力学中出现的无穷维耗散系统的渐近分析的数学背景和方法感兴趣的物理学家和工程师有用。
	
	我们的介绍基于一般和抽象的模型,涵盖了几个重要的生成无穷维耗散系统的非线性偏微分方程种类,它们包括热以及反应扩散模型、用于研究湍流现象的二维流体动力学中出现的广泛模型和具有非线性状态依赖阻尼的板波模型,我们还考虑了非线性阻尼波Kirchhoff模型和一些抛物和双曲类型的迟滞问题。
	
	本书中的大部分分析都致力于动力学的稳定性,以及将无限维系统严格简化为某些有限维结构,这些结构仅通过有限多个自由度来刻画。这些有限维结构应该引起应用型科学家的兴趣,他们追求现实的无穷维现象的数学模拟。
	
	本书包括大量的习题,正如Dan Henry著名的专著\cite{Henry81}那样,它们是本书不可或缺的一部分,它们中的大多数被策略性地放置在文本中,而不是放在一个部分的末尾,一些习题是常规的,而另外一些习题则是以“习题”形式编写的一般的评论和注解,这使得我们能够缩短叙述以及避免额外的细节。
	
	本书可以作为研究生阶段耗散动力学课程的教材。了解泛函分析和常微分方程的基本概念和事实就足以理解本书。 事实上,本书的许多部分已经在作者在Kharkov大学开设的高级本科和研究生课程中使用过。

	\section*{致谢}

	我很高兴向所有帮助我理解耗散动力学本质的同事表示感谢。我最衷心的感谢Irena Lasiecka,我们在非线性偏微分方程动力学方面进行了许多非常启发性的讨论和愉快的合作。我要感谢Alexander Rezounenko对状态依赖时滞模型的评论和富有成效的讨论。感谢我的儿子Gennadiy在绘制所有插图时的慷慨帮助,感谢我的妻子Galina对我在创作本书时长久的鼓励。我还要感谢Springer出版社的编辑们(尤其是Donna Chernyk)对这个项目的关心和值得赞赏的帮助。在自2014年春季初稿开始创作的过程中,我收到了相当多匿名审稿人的11条(十一条!)意见,感谢他们的宝贵意见和建议。
	\newline\newline
	Kharkov, Ukraine\hfill Igor Chueshov
	\newline
	2015年6月
\end{PreChapter}	
	
	